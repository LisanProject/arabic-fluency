% Options for packages loaded elsewhere
% Options for packages loaded elsewhere
\PassOptionsToPackage{unicode}{hyperref}
\PassOptionsToPackage{hyphens}{url}
\PassOptionsToPackage{dvipsnames,svgnames,x11names}{xcolor}
%
\documentclass[
  a4paper,
  DIV=11,
  numbers=noendperiod]{scrartcl}
\usepackage{xcolor}
\usepackage{amsmath,amssymb}
\setcounter{secnumdepth}{5}
\usepackage{iftex}
\ifPDFTeX
  \usepackage[T1]{fontenc}
  \usepackage[utf8]{inputenc}
  \usepackage{textcomp} % provide euro and other symbols
\else % if luatex or xetex
  \usepackage{unicode-math} % this also loads fontspec
  \defaultfontfeatures{Scale=MatchLowercase}
  \defaultfontfeatures[\rmfamily]{Ligatures=TeX,Scale=1}
\fi
\usepackage{lmodern}
\ifPDFTeX\else
  % xetex/luatex font selection
\fi
% Use upquote if available, for straight quotes in verbatim environments
\IfFileExists{upquote.sty}{\usepackage{upquote}}{}
\IfFileExists{microtype.sty}{% use microtype if available
  \usepackage[]{microtype}
  \UseMicrotypeSet[protrusion]{basicmath} % disable protrusion for tt fonts
}{}
\makeatletter
\@ifundefined{KOMAClassName}{% if non-KOMA class
  \IfFileExists{parskip.sty}{%
    \usepackage{parskip}
  }{% else
    \setlength{\parindent}{0pt}
    \setlength{\parskip}{6pt plus 2pt minus 1pt}}
}{% if KOMA class
  \KOMAoptions{parskip=half}}
\makeatother
% Make \paragraph and \subparagraph free-standing
\makeatletter
\ifx\paragraph\undefined\else
  \let\oldparagraph\paragraph
  \renewcommand{\paragraph}{
    \@ifstar
      \xxxParagraphStar
      \xxxParagraphNoStar
  }
  \newcommand{\xxxParagraphStar}[1]{\oldparagraph*{#1}\mbox{}}
  \newcommand{\xxxParagraphNoStar}[1]{\oldparagraph{#1}\mbox{}}
\fi
\ifx\subparagraph\undefined\else
  \let\oldsubparagraph\subparagraph
  \renewcommand{\subparagraph}{
    \@ifstar
      \xxxSubParagraphStar
      \xxxSubParagraphNoStar
  }
  \newcommand{\xxxSubParagraphStar}[1]{\oldsubparagraph*{#1}\mbox{}}
  \newcommand{\xxxSubParagraphNoStar}[1]{\oldsubparagraph{#1}\mbox{}}
\fi
\makeatother


\usepackage{longtable,booktabs,array}
\usepackage{calc} % for calculating minipage widths
% Correct order of tables after \paragraph or \subparagraph
\usepackage{etoolbox}
\makeatletter
\patchcmd\longtable{\par}{\if@noskipsec\mbox{}\fi\par}{}{}
\makeatother
% Allow footnotes in longtable head/foot
\IfFileExists{footnotehyper.sty}{\usepackage{footnotehyper}}{\usepackage{footnote}}
\makesavenoteenv{longtable}
\usepackage{graphicx}
\makeatletter
\newsavebox\pandoc@box
\newcommand*\pandocbounded[1]{% scales image to fit in text height/width
  \sbox\pandoc@box{#1}%
  \Gscale@div\@tempa{\textheight}{\dimexpr\ht\pandoc@box+\dp\pandoc@box\relax}%
  \Gscale@div\@tempb{\linewidth}{\wd\pandoc@box}%
  \ifdim\@tempb\p@<\@tempa\p@\let\@tempa\@tempb\fi% select the smaller of both
  \ifdim\@tempa\p@<\p@\scalebox{\@tempa}{\usebox\pandoc@box}%
  \else\usebox{\pandoc@box}%
  \fi%
}
% Set default figure placement to htbp
\def\fps@figure{htbp}
\makeatother





\setlength{\emergencystretch}{3em} % prevent overfull lines

\providecommand{\tightlist}{%
  \setlength{\itemsep}{0pt}\setlength{\parskip}{0pt}}



 


\KOMAoption{captions}{tableheading}
\makeatletter
\@ifpackageloaded{tcolorbox}{}{\usepackage[skins,breakable]{tcolorbox}}
\@ifpackageloaded{fontawesome5}{}{\usepackage{fontawesome5}}
\definecolor{quarto-callout-color}{HTML}{909090}
\definecolor{quarto-callout-note-color}{HTML}{0758E5}
\definecolor{quarto-callout-important-color}{HTML}{CC1914}
\definecolor{quarto-callout-warning-color}{HTML}{EB9113}
\definecolor{quarto-callout-tip-color}{HTML}{00A047}
\definecolor{quarto-callout-caution-color}{HTML}{FC5300}
\definecolor{quarto-callout-color-frame}{HTML}{acacac}
\definecolor{quarto-callout-note-color-frame}{HTML}{4582ec}
\definecolor{quarto-callout-important-color-frame}{HTML}{d9534f}
\definecolor{quarto-callout-warning-color-frame}{HTML}{f0ad4e}
\definecolor{quarto-callout-tip-color-frame}{HTML}{02b875}
\definecolor{quarto-callout-caution-color-frame}{HTML}{fd7e14}
\makeatother
\makeatletter
\@ifpackageloaded{caption}{}{\usepackage{caption}}
\AtBeginDocument{%
\ifdefined\contentsname
  \renewcommand*\contentsname{Table of contents}
\else
  \newcommand\contentsname{Table of contents}
\fi
\ifdefined\listfigurename
  \renewcommand*\listfigurename{List of Figures}
\else
  \newcommand\listfigurename{List of Figures}
\fi
\ifdefined\listtablename
  \renewcommand*\listtablename{List of Tables}
\else
  \newcommand\listtablename{List of Tables}
\fi
\ifdefined\figurename
  \renewcommand*\figurename{Figure}
\else
  \newcommand\figurename{Figure}
\fi
\ifdefined\tablename
  \renewcommand*\tablename{Table}
\else
  \newcommand\tablename{Table}
\fi
}
\@ifpackageloaded{float}{}{\usepackage{float}}
\floatstyle{ruled}
\@ifundefined{c@chapter}{\newfloat{codelisting}{h}{lop}}{\newfloat{codelisting}{h}{lop}[chapter]}
\floatname{codelisting}{Listing}
\newcommand*\listoflistings{\listof{codelisting}{List of Listings}}
\makeatother
\makeatletter
\makeatother
\makeatletter
\@ifpackageloaded{caption}{}{\usepackage{caption}}
\@ifpackageloaded{subcaption}{}{\usepackage{subcaption}}
\makeatother
\usepackage{bookmark}
\IfFileExists{xurl.sty}{\usepackage{xurl}}{} % add URL line breaks if available
\urlstyle{same}
\hypersetup{
  pdftitle={Arabic Fluency Roadmap},
  pdfauthor={Liban Hussein},
  colorlinks=true,
  linkcolor={blue},
  filecolor={Maroon},
  citecolor={Blue},
  urlcolor={Blue},
  pdfcreator={LaTeX via pandoc}}


\title{Arabic Fluency Roadmap}
\author{Liban Hussein}
\date{2025-06-16}
\begin{document}
\maketitle

\renewcommand*\contentsname{Table of contents}
{
\hypersetup{linkcolor=}
\setcounter{tocdepth}{3}
\tableofcontents
}

\section{Introduction}\label{sec-intro}

This document is a comprehensive companion to an 8-month Arabic language
learning journey based on the textbook
\href{https://attakallum.com/attakallum-online/login}{\textbf{\emph{At-Takallum:
A Comprehensive Modern Arabic Course}}}.

It serves as a central hub for:

\begin{itemize}
\tightlist
\item
  \textbf{Detailed study notes}\\
\item
  \textbf{Thematic and contextual vocabulary lists}\\
\item
  \textbf{Practice exercises} for grammar, reading, writing, listening,
  and speaking\\
\item
  \textbf{Supplementary resources} for cultural and religious
  enrichment\\
\item
  \textbf{Tracking overall progression} across all learning levels
\end{itemize}

The learning journey is divided into three stages, each aligned with a
CEFR proficiency level and structured around thematic chapters designed
to gradually build language skills in context.

\begin{center}\rule{0.5\linewidth}{0.5pt}\end{center}

\subsection{Elementary Level (A1
Proficiency)}\label{elementary-level-a1-proficiency}

The \textbf{Elementary Book} consists of 8 foundational chapters focused
on everyday communication and basic vocabulary:

\begin{enumerate}
\def\labelenumi{\arabic{enumi}.}
\tightlist
\item
  \textbf{Greetings and Introductions}\\
\item
  \textbf{Family and Relationships}\\
\item
  \textbf{School and Studies}\\
\item
  \textbf{Food and Eating Habits}\\
\item
  \textbf{Telling Time and Discussing Prices}\\
\item
  \textbf{Holidays and Vacations}\\
\item
  \textbf{Daily Routines and Activities}\\
\item
  \textbf{Weather and Clothing}
\end{enumerate}

\begin{center}\rule{0.5\linewidth}{0.5pt}\end{center}

\subsection{Pre-Intermediate Level (B1
Proficiency)}\label{pre-intermediate-level-b1-proficiency}

The \textbf{Pre-Intermediate Book} continues with more complex sentence
structures and a wider vocabulary, divided into the following 8 thematic
units:

\begin{enumerate}
\def\labelenumi{\arabic{enumi}.}
\tightlist
\item
  \textbf{Exploring Cairo and Transportation}\\
\item
  \textbf{Housing and Daily Living}\\
\item
  \textbf{Health and Well-being}\\
\item
  \textbf{Shopping and Markets}\\
\item
  \textbf{Jobs and the Workplace}\\
\item
  \textbf{Tourist Attractions and Historical Sites}\\
\item
  \textbf{Sports and Hobbies}\\
\item
  \textbf{Past Memories and Storytelling}
\end{enumerate}

\begin{center}\rule{0.5\linewidth}{0.5pt}\end{center}

\subsection{Intermediate Level (B2
Proficiency)}\label{intermediate-level-b2-proficiency}

The \textbf{Intermediate Book} builds toward fluency, focusing on
abstract topics and culturally rich discussions:

\begin{enumerate}
\def\labelenumi{\arabic{enumi}.}
\tightlist
\item
  \textbf{Describing People and Character Traits}\\
\item
  \textbf{Religious Celebrations and Eid Traditions}\\
\item
  \textbf{Humor and Storytelling}\\
\item
  \textbf{Nature and Environmental Issues}\\
\item
  \textbf{World Cultures and Social Traditions}\\
\item
  \textbf{Arabic Proverbs, Sayings, and Wisdom}\\
\item
  \textbf{Famous Historical Figures}\\
\item
  \textbf{The Role of Education and Learning}
\end{enumerate}

\begin{center}\rule{0.5\linewidth}{0.5pt}\end{center}

\subsubsection{Materials and Resources}\label{materials-and-resources}

\begin{itemize}
\tightlist
\item
  Supplementary notes, vocabulary decks, and grammar exercises\\
\item
  Cultural and religious content to enrich learning context
\item
  Translation practice using classical texts
\end{itemize}

\begin{tcolorbox}[enhanced jigsaw, opacitybacktitle=0.6, colbacktitle=quarto-callout-important-color!10!white, arc=.35mm, opacityback=0, colback=white, rightrule=.15mm, breakable, titlerule=0mm, colframe=quarto-callout-important-color-frame, title=\textcolor{quarto-callout-important-color}{\faExclamation}\hspace{0.5em}{Note}, leftrule=.75mm, left=2mm, bottomtitle=1mm, toptitle=1mm, bottomrule=.15mm, toprule=.15mm, coltitle=black]

This document will evolve over time to reflect ongoing progress,
feedback, and new materials. Contributions and suggestions are welcome.

\end{tcolorbox}

\section{التَّعَارُف -- Introducing Yourself}\label{unit1-intro}

Making a great first impression starts with the basics: greetings,
names, nationalities, and simple questions. This unit introduces
essential vocabulary and expressions for everyday social interactions
--- from saying hello to telling someone where you're from.

Use the table below to familiarize yourself with the most common phrases
used in first encounters.

\begin{center}\rule{0.5\linewidth}{0.5pt}\end{center}

\begin{tcolorbox}[enhanced jigsaw, opacitybacktitle=0.6, colbacktitle=quarto-callout-tip-color!10!white, arc=.35mm, opacityback=0, colback=white, rightrule=.15mm, breakable, titlerule=0mm, colframe=quarto-callout-tip-color-frame, title=\textcolor{quarto-callout-tip-color}{\faLightbulb}\hspace{0.5em}{Watch This!}, leftrule=.75mm, left=2mm, bottomtitle=1mm, toptitle=1mm, bottomrule=.15mm, toprule=.15mm, coltitle=black]

Here's a helpful video that provides some pretty solid techniques for
memorizing new vocabulary:\\
\href{https://www.youtube.com/watch?v=TZYOScgVc3g&t}{7 Insanely
Effective Techniques to Memorize Vocabulary in a New Language}

\end{tcolorbox}

\subsection{Key Vocabulary and
Phrases}\label{key-vocabulary-and-phrases}

\begin{longtable}[]{@{}
  >{\raggedright\arraybackslash}p{(\linewidth - 2\tabcolsep) * \real{0.4375}}
  >{\raggedright\arraybackslash}p{(\linewidth - 2\tabcolsep) * \real{0.5625}}@{}}
\toprule\noalign{}
\begin{minipage}[b]{\linewidth}\raggedright
English
\end{minipage} & \begin{minipage}[b]{\linewidth}\raggedright
Arabic
\end{minipage} \\
\midrule\noalign{}
\endhead
\bottomrule\noalign{}
\endlastfoot
Peace be upon you! & السَّلَامُ عَلَيْكُمْ \\
And peace be upon you! & وَعَلَيْكُمُ السَّلَامُ \\
Good morning! & صَبَاحُ الْخَيْرِ / صَبَاحُ النُّورِ \\
How are you? & كَيْفَ حَالُكَ؟ / كَيْفَ الْحَالُ؟ \\
Fine, al-Hamdulillah & بِخَيْر / بِخَيْر وَالْحَمْدُ لِلّٰه \\
Welcome! & أَهْلًا وَسَهْلًا / مَرْحَبًا \\
Welcome to you & أَهْلًا بِكَ / مَرْحَبًا بِكَ \\
What is your name? & مَا اسْمُكَ؟ \\
My name is \ldots{} & اِسْمِي \ldots{} \\
Where are you from? & مِنْ أَيْنَ أَنْتَ؟ \\
I am Egyptian & أَنَا مِصْرِيٌّ \\
I am from Egypt & أَنَا مِنْ مِصْرَ \\
Good evening! & مَسَاءُ الْخَيْرِ / مَسَاءُ النُّورِ \\
Nice to meet you & تَشَرَّفْنَا / فُرْصَة طَيِّبَة / سَعِيدَة \\
Goodbye / Farewell & فِي أَمَانِ اللّٰه / مَعَ السَّلَامَة / إِلَى اللِّقَاءِ \\
Job / Profession / Work & وَظِيفَة / مِهْنَة / عَمَل \\
Student & طَالِب / تِلْمِيذ \\
Teacher & مُعَلِّم / مُدَرِّس / أُسْتَاذ \\
Translator & مُتَرْجِم \\
Worker / Laborer & عَامِل \\
Friend / Colleague & صَدِيقَة / زَمِيلَة \\
Carpenter & نَجَّار \\
Retired & مُتَقَاعِد \\
Journalist & صَحَفِيّ \\
Accountant & مُحَاسِب \\
Mobile phone & هَاتِف مَحْمُول \\
Landline phone & هَاتِف ثَابِت / أَرْضِي \\
Email & بَرِيد إِلِكْتِرُونِي \\
Address & عُنْوَان \\
Unemployed & عَاطِل \\
He works & يَعْمَلُ \\
\end{longtable}

\begin{center}\rule{0.5\linewidth}{0.5pt}\end{center}

\begin{tcolorbox}[enhanced jigsaw, opacitybacktitle=0.6, colbacktitle=quarto-callout-note-color!10!white, arc=.35mm, opacityback=0, colback=white, rightrule=.15mm, breakable, titlerule=0mm, colframe=quarto-callout-note-color-frame, title=\textcolor{quarto-callout-note-color}{\faInfo}\hspace{0.5em}{Try This!}, leftrule=.75mm, left=2mm, bottomtitle=1mm, toptitle=1mm, bottomrule=.15mm, toprule=.15mm, coltitle=black]

Try translating the following English phrases into Arabic using only the
vocabulary from Chapter 1.

\begin{enumerate}
\def\labelenumi{\arabic{enumi}.}
\tightlist
\item
  ``My name is \ldots{} and I am from \ldots{}''
\item
  ``What is your name {[}masculine/feminine{]}?''
\item
  ``I am Egyptian.''
\item
  ``I am from Egypt.''
\item
  ``Where are you from?''
\item
  ``What is your profession?''
\item
  ``I am a teacher.''
\item
  ``I am fine, thanks.''
\item
  ``Good morning!''
\item
  ``Goodbye! See you later!''
\end{enumerate}

\end{tcolorbox}

\subsection{Singular Pronouns}\label{singular-pronouns}

In Arabic, personal pronouns change depending on gender and number. Here
are the \textbf{singular pronouns}:

\begin{longtable}[]{@{}lll@{}}
\toprule\noalign{}
English & Arabic & Pronunciation \\
\midrule\noalign{}
\endhead
\bottomrule\noalign{}
\endlastfoot
I & أَنَا & \emph{anā} \\
You (masc.) & أَنْتَ & \emph{anta} \\
You (fem.) & أَنْتِ & \emph{anti} \\
He & هُوَ & \emph{huwa} \\
She & هِيَ & \emph{hiya} \\
\end{longtable}

\begin{tcolorbox}[enhanced jigsaw, opacitybacktitle=0.6, colbacktitle=quarto-callout-tip-color!10!white, arc=.35mm, opacityback=0, colback=white, rightrule=.15mm, breakable, titlerule=0mm, colframe=quarto-callout-tip-color-frame, title=\textcolor{quarto-callout-tip-color}{\faLightbulb}\hspace{0.5em}{Possessive Endings Tip!}, leftrule=.75mm, left=2mm, bottomtitle=1mm, toptitle=1mm, bottomrule=.15mm, toprule=.15mm, coltitle=black]

To express \textbf{possession} (like \emph{my book}, \emph{your phone},
\emph{his teacher}), Arabic uses \textbf{suffixes} attached to the end
of the noun. Here are the common \textbf{possessive endings} for
singular pronouns:

\begin{itemize}
\tightlist
\item
  \textbf{ي} (ـي) for \textbf{أَنَا} → \emph{كِتَابِي} (my book)\\
\item
  \textbf{كَ} (ـكَ) for \textbf{أَنْتَ} → \emph{كِتَابُكَ} (your book, masc.)\\
\item
  \textbf{كِ} (ـكِ) for \textbf{أَنْتِ} → \emph{كِتَابُكِ} (your book, fem.)\\
\item
  \textbf{هُ} (ـهُ) for \textbf{هُوَ} → \emph{كِتَابُهُ} (his book)\\
\item
  \textbf{هَا} (ـهَا) for \textbf{هِيَ} → \emph{كِتَابُهَا} (her book)
\end{itemize}

These suffixes are directly attached to nouns to show ownership.

\end{tcolorbox}

\subsection{Nationalities}\label{nationalities}

Nationalities in Arabic are based on the name of the country and are
modified by adding masculine or feminine endings. These follow regular
patterns and are important for introducing yourself or describing
others.

\begin{tcolorbox}[enhanced jigsaw, opacitybacktitle=0.6, colbacktitle=quarto-callout-tip-color!10!white, arc=.35mm, opacityback=0, colback=white, rightrule=.15mm, breakable, titlerule=0mm, colframe=quarto-callout-tip-color-frame, title=\textcolor{quarto-callout-tip-color}{\faLightbulb}\hspace{0.5em}{Country Roots \& Nationalities}, leftrule=.75mm, left=2mm, bottomtitle=1mm, toptitle=1mm, bottomrule=.15mm, toprule=.15mm, coltitle=black]

\begin{longtable}[]{@{}lll@{}}
\toprule\noalign{}
Country & Root (Base Word) & Nationality (Masc. / Fem.) \\
\midrule\noalign{}
\endhead
\bottomrule\noalign{}
\endlastfoot
Egypt & مِصْر & مِصْرِيّ / مِصْرِيَّة \\
Somalia & الصُّومَال & صُومَالِيّ / صُومَالِيَّة \\
America & أَمْرِيكَا & أَمْرِيكِيّ / أَمْرِيكِيَّة \\
\end{longtable}

\subsubsection{Examples:}\label{examples}

\begin{itemize}
\tightlist
\item
  \textbf{أَنَا مِصْرِيّ} -- I am Egyptian (male)\\
\item
  \textbf{أَنَا مِصْرِيَّة} -- I am Egyptian (female)\\
\item
  \textbf{هُوَ صُومَالِيّ} -- He is Somali\\
\item
  \textbf{هِيَ أَمْرِيكِيَّة} -- She is American
\end{itemize}

\begin{quote}
Note: The masculine form usually ends in \textbf{ـِيّ}, and the feminine
form ends in \textbf{ـِيَّة}.
\end{quote}

\end{tcolorbox}

\subsubsection{Practice Questions}\label{practice-questions}

Try answering the following using the correct nationality form:

\begin{enumerate}
\def\labelenumi{\arabic{enumi}.}
\tightlist
\item
  مَا جِنْسِيَّتُكَ؟ (What is your nationality? --- to a male)\\
\item
  مَا جِنْسِيَّتُكِ؟ (What is your nationality? --- to a female)\\
\item
  Translate:

  \begin{enumerate}
  \def\labelenumii{\alph{enumii}.}
  \tightlist
  \item
    I am American (female)\\
  \item
    He is Egyptian\\
  \item
    She is Somali
  \end{enumerate}
\end{enumerate}

\begin{center}\rule{0.5\linewidth}{0.5pt}\end{center}

\begin{tcolorbox}[enhanced jigsaw, opacitybacktitle=0.6, colbacktitle=quarto-callout-caution-color!10!white, arc=.35mm, opacityback=0, colback=white, rightrule=.15mm, breakable, titlerule=0mm, colframe=quarto-callout-caution-color-frame, title=\textcolor{quarto-callout-caution-color}{\faFire}\hspace{0.5em}{Click to Practice a Conversation}, leftrule=.75mm, left=2mm, bottomtitle=1mm, toptitle=1mm, bottomrule=.15mm, toprule=.15mm, coltitle=black]

\subsection{Practice: A Conversation Between Two
Friends}\label{practice-a-conversation-between-two-friends}

\textbf{Friend 1:} Hello!\\
\textbf{Friend 2:} Hello and welcome!\\
\textbf{Friend 1:} How are you?\\
\textbf{Friend 2:} I'm fine, thank God. And you?\\
\textbf{Friend 1:} I'm fine. What is your name?\\
\textbf{Friend 2:} My name is Ahmed. What is your name?\\
\textbf{Friend 1:} My name is Sarah. Where are you from?\\
\textbf{Friend 2:} I am from Egypt. And you?\\
\textbf{Friend 1:} I am from America. What is your job?\\
\textbf{Friend 2:} I am a teacher. And you?\\
\textbf{Friend 1:} I am a student. What is your address?\\
\textbf{Friend 2:} My address is 12 Cairo Street. What is your
address?\\
\textbf{Friend 1:} My address is 8 Nile Avenue. What is your phone
number?\\
\textbf{Friend 2:} My mobile number is 010-1234-5678. What is your
number?\\
\textbf{Friend 1:} My phone number is 011-9876-5432. What is your
email?\\
\textbf{Friend 2:} My email is ahmed@gmail.com. And yours?\\
\textbf{Friend 1:} My email is sarah@email.com.\\
\textbf{Friend 2:} Nice to meet you!\\
\textbf{Friend 1:} Nice to meet you too!

\end{tcolorbox}

\subsection{The Present Simple Tense With Singular
Pronouns}\label{the-present-simple-tense-with-singular-pronouns}

In Arabic, verbs in the present tense are based on three-letter roots.
Let's start with the verb \textbf{عَمِلَ} (to work) and see how it changes
with singular pronouns in the \textbf{present tense}.

\begin{longtable}[]{@{}
  >{\raggedright\arraybackslash}p{(\linewidth - 4\tabcolsep) * \real{0.1548}}
  >{\raggedright\arraybackslash}p{(\linewidth - 4\tabcolsep) * \real{0.2381}}
  >{\raggedright\arraybackslash}p{(\linewidth - 4\tabcolsep) * \real{0.6071}}@{}}
\toprule\noalign{}
\begin{minipage}[b]{\linewidth}\raggedright
Pronoun
\end{minipage} & \begin{minipage}[b]{\linewidth}\raggedright
Present Tense Verb
\end{minipage} & \begin{minipage}[b]{\linewidth}\raggedright
Example Sentence
\end{minipage} \\
\midrule\noalign{}
\endhead
\bottomrule\noalign{}
\endlastfoot
I & أَعْمَلُ & أَنَا أَعْمَلُ مُتَرْجِمًا -- I work as a translator \\
You (m) & تَعْمَلُ & أَنْتَ تَعْمَلُ مُتَرْجِمًا -- You (m) work as a translator \\
You (f) & تَعْمَلِينَ & أَنْتِ تَعْمَلِينَ مُتَرْجِمَةً -- You (f) work as a translator \\
He & يَعْمَلُ & هُوَ يَعْمَلُ مُتَرْجِمًا -- He works as a translator \\
She & تَعْمَلُ & هِيَ تَعْمَلُ مُتَرْجِمَةً -- She works as a translator \\
\end{longtable}

\begin{quote}
Note: Feminine job titles and pronouns require feminine verb forms and
endings, like \textbf{ـينَ} and \textbf{ـة}.
\end{quote}

\begin{center}\rule{0.5\linewidth}{0.5pt}\end{center}

\begin{tcolorbox}[enhanced jigsaw, opacitybacktitle=0.6, colbacktitle=quarto-callout-caution-color!10!white, arc=.35mm, opacityback=0, colback=white, rightrule=.15mm, breakable, titlerule=0mm, colframe=quarto-callout-caution-color-frame, title=\textcolor{quarto-callout-caution-color}{\faFire}\hspace{0.5em}{Click to Practice with More Root Verbs}, leftrule=.75mm, left=2mm, bottomtitle=1mm, toptitle=1mm, bottomrule=.15mm, toprule=.15mm, coltitle=black]

\subsubsection{Practice with These Root
Verbs:}\label{practice-with-these-root-verbs}

\begin{longtable}[]{@{}lll@{}}
\toprule\noalign{}
Verb Root & Meaning & Present Tense (He) \\
\midrule\noalign{}
\endhead
\bottomrule\noalign{}
\endlastfoot
عَمِلَ & to work & يَعْمَلُ \\
كَتَبَ & to write & يَكْتُبُ \\
قَرَأَ & to read & يَقْرَأُ \\
نَظَرَ & to look & يَنْظُرُ \\
فَهِمَ & to understand & يَفْهَمُ \\
\end{longtable}

\subsubsection{Practice Prompts:}\label{practice-prompts}

Try writing these for each singular pronoun (\textbf{أنا، أنتَ، أنتِ، هو،
هي}):

\begin{enumerate}
\def\labelenumi{\arabic{enumi}.}
\tightlist
\item
  ``\_\_\_ writes an email.''\\
\item
  ``\_\_\_ reads the book.''\\
\item
  ``\_\_\_ looks at the address.''\\
\item
  ``\_\_\_ understands the lesson.''\\
\item
  ``\_\_\_ works as a teacher.''
\end{enumerate}

Mix masculine and feminine forms!

\end{tcolorbox}

\begin{center}\rule{0.5\linewidth}{0.5pt}\end{center}

\subsection{Quick Review Section}\label{quick-review-section}

\begin{tcolorbox}[enhanced jigsaw, opacitybacktitle=0.6, colbacktitle=quarto-callout-note-color!10!white, arc=.35mm, opacityback=0, colback=white, rightrule=.15mm, breakable, titlerule=0mm, colframe=quarto-callout-note-color-frame, title=\textcolor{quarto-callout-note-color}{\faInfo}\hspace{0.5em}{Review Checklist}, leftrule=.75mm, left=2mm, bottomtitle=1mm, toptitle=1mm, bottomrule=.15mm, toprule=.15mm, coltitle=black]

Before moving on, make sure you can:

\begin{itemize}
\tightlist
\item[$\square$]
  Introduce yourself in Arabic\\
\item[$\square$]
  Ask and answer where someone is from\\
\item[$\square$]
  Use singular pronouns correctly\\
\item[$\square$]
  Identify masculine and feminine nationalities\\
\item[$\square$]
  Form simple present-tense sentences with key verbs\\
\item[$\square$]
  Ask for basic personal info: name, address, phone, job
\end{itemize}

\end{tcolorbox}

\begin{tcolorbox}[enhanced jigsaw, opacitybacktitle=0.6, colbacktitle=quarto-callout-tip-color!10!white, arc=.35mm, opacityback=0, colback=white, rightrule=.15mm, breakable, titlerule=0mm, colframe=quarto-callout-tip-color-frame, title=\textcolor{quarto-callout-tip-color}{\faLightbulb}\hspace{0.5em}{A Scholar's Reflection}, leftrule=.75mm, left=2mm, bottomtitle=1mm, toptitle=1mm, bottomrule=.15mm, toprule=.15mm, coltitle=black]

\textbf{``اللغة العربية هي من أغنى اللغات وأوسعها، وهي بحرٌ لا ساحل
له.''}\\
\emph{``The Arabic language is among the richest and most expansive
languages; it is a sea without a shore.''}\\
--- \textbf{Ibn Jinnī}, famed linguist and grammarian

\end{tcolorbox}

\section{الْعَائِلَة -- The Family}\label{unit2-family}

Family is the foundation of many conversations in Arabic. In this unit,
you'll learn how to introduce your family members, describe
relationships, and talk about where people live and what they do.

\begin{center}\rule{0.5\linewidth}{0.5pt}\end{center}

\subsection{Key Vocabulary and
Phrases}\label{key-vocabulary-and-phrases-1}

\begin{longtable}[]{@{}ll@{}}
\toprule\noalign{}
English & Arabic \\
\midrule\noalign{}
\endhead
\bottomrule\noalign{}
\endlastfoot
Family / Household & أُسْرَة / عَائِلَة \\
Grandfather & جَدّ \\
Grandmother & جَدَّة \\
Father / Parent & أَب / وَالِد \\
Mother / Parent & أُمّ / وَالِدَة \\
Brother & أَخ \\
Sister & أُخْت \\
Son / Daughter & اِبْن / اِبْنَة \\
Boy / Girl & وَلَد / بِنْت \\
Grandson / Granddaughter & حَفِيد / حَفِيدَة \\
Twin brother & أَخ تَوْأَم \\
Paternal uncle & عَمّ \\
Maternal uncle & خَال \\
Paternal aunt & عَمَّة \\
Maternal aunt & خَالَة \\
Husband / Wife & زَوْج / زَوْجَة \\
Relative / Kin & قَرِيب / أَقَارِب \\
Father-in-law / Mother-in-law & حَم / حَمَاة \\
Married & مُتَزَوِّج / مُتَزَوِّجَة \\
Fiancé / Fiancée & خَاطِب / مَخْطُوبَة \\
Bachelor (unmarried) & عَزَب \\
Miss / Young lady & آنِسَة \\
Surname / Title & لَقَب / أَلْقَاب \\
\end{longtable}

\begin{center}\rule{0.5\linewidth}{0.5pt}\end{center}

\subsection{Talking About Daily Life}\label{talking-about-daily-life}

\begin{longtable}[]{@{}ll@{}}
\toprule\noalign{}
English & Arabic \\
\midrule\noalign{}
\endhead
\bottomrule\noalign{}
\endlastfoot
He lives in\ldots{} & يَسْكُنُ فِي\ldots{} \\
She studies & تَدْرُس \\
He loves / likes & يُحِبُّ \\
He comments & يُعَلِّق \\
He receives / meets & يَسْتَقْبِل \\
He prepares / makes ready & يُرَتِّب \\
She cooks & تَطْبُخ \\
He sits / sits down & يَجْلِس \\
\end{longtable}

\begin{center}\rule{0.5\linewidth}{0.5pt}\end{center}

\subsection{Practice Phrases}\label{practice-phrases}

\begin{tcolorbox}[enhanced jigsaw, opacitybacktitle=0.6, colbacktitle=quarto-callout-note-color!10!white, arc=.35mm, opacityback=0, colback=white, rightrule=.15mm, breakable, titlerule=0mm, colframe=quarto-callout-note-color-frame, title=\textcolor{quarto-callout-note-color}{\faInfo}\hspace{0.5em}{Try This!}, leftrule=.75mm, left=2mm, bottomtitle=1mm, toptitle=1mm, bottomrule=.15mm, toprule=.15mm, coltitle=black]

Translate the following into Arabic using the vocabulary above:

\begin{enumerate}
\def\labelenumi{\arabic{enumi}.}
\tightlist
\item
  My mother and father live in Cairo.
\item
  He is my maternal uncle.
\item
  My sister is a student.
\item
  She loves her grandmother.
\item
  I am married.
\item
  He is a bachelor.
\item
  This is my wife and these are our children.
\item
  Where does your grandfather live?
\item
  She studies in the university.
\item
  My aunt (paternal) cooks delicious food.
\end{enumerate}

\end{tcolorbox}

\subsection{Special Occasions \& Home
Life}\label{special-occasions-home-life}

\begin{longtable}[]{@{}ll@{}}
\toprule\noalign{}
English & Arabic \\
\midrule\noalign{}
\endhead
\bottomrule\noalign{}
\endlastfoot
Bride / Bridegroom & عَرُوس / عَرِيس \\
Food & طَعَام \\
Road / Street & طَرِيق \\
Decoration & زِينَة \\
Guests & ضُيُوف / ضَيْف \\
Salon / Reception Room & صَالُون \\
Furniture & أَثَاث \\
\end{longtable}

\begin{center}\rule{0.5\linewidth}{0.5pt}\end{center}

\begin{tcolorbox}[enhanced jigsaw, opacitybacktitle=0.6, colbacktitle=quarto-callout-caution-color!10!white, arc=.35mm, opacityback=0, colback=white, rightrule=.15mm, breakable, titlerule=0mm, colframe=quarto-callout-caution-color-frame, title=\textcolor{quarto-callout-caution-color}{\faFire}\hspace{0.5em}{Click to Practice a Dialogue}, leftrule=.75mm, left=2mm, bottomtitle=1mm, toptitle=1mm, bottomrule=.15mm, toprule=.15mm, coltitle=black]

\subsection{Practice: Talking About
Family}\label{practice-talking-about-family}

\textbf{Ahmed:} Hello! Who is this?\\
\textbf{Fatimah:} This is my father and mother.\\
\textbf{Ahmed:} What is your father's name?\\
\textbf{Fatimah:} His name is Yusuf.\\
\textbf{Ahmed:} Does he work?\\
\textbf{Fatimah:} Yes, he works as a teacher.\\
\textbf{Ahmed:} And your mother?\\
\textbf{Fatimah:} She cooks and takes care of the house.\\
\textbf{Ahmed:} Do you have any siblings?\\
\textbf{Fatimah:} Yes, I have one brother and two sisters.\\
\textbf{Ahmed:} Where do you live?\\
\textbf{Fatimah:} We live in Tripoli.

\end{tcolorbox}

\begin{center}\rule{0.5\linewidth}{0.5pt}\end{center}

\subsection{Verb Practice With the Family
Theme}\label{verb-practice-with-the-family-theme}

Try conjugating and writing present-tense sentences using these verbs:
\textbf{يَسْكُن، يُحِب، يُرَتِّب، تَدْرُس، تَطْبُخ}

\begin{longtable}[]{@{}ll@{}}
\toprule\noalign{}
Pronoun & Example Verb + Family Context \\
\midrule\noalign{}
\endhead
\bottomrule\noalign{}
\endlastfoot
أَنَا & أَنَا أُحِبُ جَدَّتِي -- I love my grandmother \\
أَنْتَ & أَنْتَ تَسْكُنُ مَعَ أُسْرَتِكَ -- You live with your family \\
أَنْتِ & أَنْتِ تَدْرُسِينَ فِي الْجَامِعَة -- You (f) study at the university \\
هُوَ & هُوَ يُرَتِّبُ الْغُرْفَة -- He arranges the room \\
هِيَ & هِيَ تَطْبُخُ الطَّعَام -- She cooks the food \\
\end{longtable}

\begin{center}\rule{0.5\linewidth}{0.5pt}\end{center}

\begin{tcolorbox}[enhanced jigsaw, opacitybacktitle=0.6, colbacktitle=quarto-callout-note-color!10!white, arc=.35mm, opacityback=0, colback=white, rightrule=.15mm, breakable, titlerule=0mm, colframe=quarto-callout-note-color-frame, title=\textcolor{quarto-callout-note-color}{\faInfo}\hspace{0.5em}{Review Checklist}, leftrule=.75mm, left=2mm, bottomtitle=1mm, toptitle=1mm, bottomrule=.15mm, toprule=.15mm, coltitle=black]

\begin{itemize}
\tightlist
\item[$\square$]
  Identify and name family members\\
\item[$\square$]
  Use verbs in present tense to describe family life\\
\item[$\square$]
  Talk about marital status and relatives\\
\item[$\square$]
  Describe who lives where\\
\item[$\square$]
  Use feminine and masculine forms correctly
\end{itemize}

\end{tcolorbox}

\section{الدِّرَاسَة -- Studying}\label{unit3-studying}

Education is a major part of life across cultures, and in Arabic,
knowing how to talk about school, subjects, and study routines helps you
connect and express yourself in academic and personal settings.

\begin{center}\rule{0.5\linewidth}{0.5pt}\end{center}

\subsection{Key Vocabulary and
Phrases}\label{key-vocabulary-and-phrases-2}

\begin{longtable}[]{@{}ll@{}}
\toprule\noalign{}
English & Arabic \\
\midrule\noalign{}
\endhead
\bottomrule\noalign{}
\endlastfoot
Class, grade & صَفّ \\
Display screen & شَاشَةُ عَرْض \\
Eraser & مِمْحَاة \\
Wastebasket & سَلَّةُ مُهْمَلَات \\
School & مَدْرَسَة \\
Bookshelf & مَكْتَبَة \\
Dictionary & قَامُوس / مُعْجَم \\
Absent / Present & غَائِب × حَاضِر \\
Page & صَفْحَة \\
Behind / At the back & خَلْف / وَرَاء \\
Between & بَيْن \\
Beside & بِجَانِب \\
Under & تَحْت / أَسْفَل \\
To the right & عَنْ يَمِين \\
To the left & عَنْ يَسَار \\
University & جَامِعَة \\
Faculty & كُلِّيَّة \\
Good / Bad & جَيِّد × سَيِّئ \\
Arabic language & اللُّغَة العَرَبِيَّة \\
English language & اللُّغَة الإِنْجِلِيزِيَّة \\
Vacation / Holiday & عُطْلَة \\
\end{longtable}

\begin{center}\rule{0.5\linewidth}{0.5pt}\end{center}

\subsection{Verbs and Phrases for the
Classroom}\label{verbs-and-phrases-for-the-classroom}

\begin{longtable}[]{@{}ll@{}}
\toprule\noalign{}
English & Arabic \\
\midrule\noalign{}
\endhead
\bottomrule\noalign{}
\endlastfoot
He studies & يَدْرُس \\
He thinks & يُفَكِّر \\
He guesses & يَظُنّ \\
He starts & يَبْدَأ \\
He finishes & يَنْتَهِي \\
He opens / He closes & يَفْتَح × يُغْلِق \\
Raise / Lower & اِرْفَع × اِخْفِض \\
Take / Here you go & خُذ / تَفَضَّل \\
Let me have & هَات \\
Lesson / Lecture & دَرْس / مُحَاضَرَة \\
Course / Subject & مَادَّة (ج) مَوَادّ \\
Idea / Opinion & فِكْرَة / رَأْي \\
Psychology & عِلْم النَّفْس \\
Jurisprudence & الْفِقْه \\
Qur'an explanation & التَّفْسِير \\
Hadith & الْحَدِيث \\
Recitation rules & التَّجْوِيد \\
History & التَّارِيخ \\
Geography & الْجُغْرَافْيَا \\
\end{longtable}

\begin{center}\rule{0.5\linewidth}{0.5pt}\end{center}

\subsection{Practice Phrases}\label{practice-phrases-1}

\begin{tcolorbox}[enhanced jigsaw, opacitybacktitle=0.6, colbacktitle=quarto-callout-note-color!10!white, arc=.35mm, opacityback=0, colback=white, rightrule=.15mm, breakable, titlerule=0mm, colframe=quarto-callout-note-color-frame, title=\textcolor{quarto-callout-note-color}{\faInfo}\hspace{0.5em}{Try This!}, leftrule=.75mm, left=2mm, bottomtitle=1mm, toptitle=1mm, bottomrule=.15mm, toprule=.15mm, coltitle=black]

Translate the following into Arabic using the vocabulary above:

\begin{enumerate}
\def\labelenumi{\arabic{enumi}.}
\tightlist
\item
  I study psychology and jurisprudence.\\
\item
  The dictionary is beside the bookshelf.\\
\item
  Open the book to page ten.\\
\item
  She is present today.\\
\item
  My course is very interesting.\\
\item
  He finishes his lesson at two o'clock.\\
\item
  The subject is difficult but useful.\\
\item
  Where is the eraser?\\
\item
  We are studying the Arabic language.\\
\item
  The classroom is to the right.
\end{enumerate}

\end{tcolorbox}

\begin{center}\rule{0.5\linewidth}{0.5pt}\end{center}

\begin{tcolorbox}[enhanced jigsaw, opacitybacktitle=0.6, colbacktitle=quarto-callout-caution-color!10!white, arc=.35mm, opacityback=0, colback=white, rightrule=.15mm, breakable, titlerule=0mm, colframe=quarto-callout-caution-color-frame, title=\textcolor{quarto-callout-caution-color}{\faFire}\hspace{0.5em}{Click to Practice a Dialogue}, leftrule=.75mm, left=2mm, bottomtitle=1mm, toptitle=1mm, bottomrule=.15mm, toprule=.15mm, coltitle=black]

\subsection{Practice: In the Classroom}\label{practice-in-the-classroom}

\textbf{Zayd:} Are you studying today?\\
\textbf{Nour:} Yes, I am studying history and psychology.\\
\textbf{Zayd:} Is the teacher present?\\
\textbf{Nour:} No, he is absent today.\\
\textbf{Zayd:} Where is your book?\\
\textbf{Nour:} It's on the bookshelf, beside the dictionary.\\
\textbf{Zayd:} Do you like this course?\\
\textbf{Nour:} Yes, it's good and interesting.

\end{tcolorbox}

\begin{center}\rule{0.5\linewidth}{0.5pt}\end{center}

\subsection{Verb Practice With the Studying
Theme}\label{verb-practice-with-the-studying-theme}

\begin{longtable}[]{@{}
  >{\raggedright\arraybackslash}p{(\linewidth - 2\tabcolsep) * \real{0.1831}}
  >{\raggedright\arraybackslash}p{(\linewidth - 2\tabcolsep) * \real{0.8169}}@{}}
\toprule\noalign{}
\begin{minipage}[b]{\linewidth}\raggedright
Pronoun
\end{minipage} & \begin{minipage}[b]{\linewidth}\raggedright
Example Verb + Classroom Context
\end{minipage} \\
\midrule\noalign{}
\endhead
\bottomrule\noalign{}
\endlastfoot
أَنَا & أَنَا أَدْرُسُ التَّفْسِير -- I study tafsīr \\
أَنْتَ & أَنْتَ تَفْتَحُ الكِتَاب -- You open the book \\
أَنْتِ & أَنْتِ تَفْهَمِينَ الدَّرْس -- You (f) understand the lesson \\
هُوَ & هُوَ يَبْدَأُ الدَّرْس -- He starts the lesson \\
هِيَ & هِيَ تُفَكِّرُ فِي الْجَوَاب -- She is thinking of the answer \\
\end{longtable}

\begin{center}\rule{0.5\linewidth}{0.5pt}\end{center}

\begin{tcolorbox}[enhanced jigsaw, opacitybacktitle=0.6, colbacktitle=quarto-callout-tip-color!10!white, arc=.35mm, opacityback=0, colback=white, rightrule=.15mm, breakable, titlerule=0mm, colframe=quarto-callout-tip-color-frame, title=\textcolor{quarto-callout-tip-color}{\faLightbulb}\hspace{0.5em}{A Scholar's Reflection}, leftrule=.75mm, left=2mm, bottomtitle=1mm, toptitle=1mm, bottomrule=.15mm, toprule=.15mm, coltitle=black]

\textbf{``أَخي لَنْ تَنَالَ الْعِلْمَ إِلَّا بِسِتَّةٍ\\
سَأُنْبِيكَ عَنْ تَفْصِيلِهَا بِبَيَانِ\\
ذَكَاءٍ، وَحِرْصٍ، وَاجْتِهَادٍ، وَبُلْغَةٍ\\
وَصُحْبَةِ أُسْتَاذٍ، وَطُولِ زَمَانِ''}\\
\emph{``My brother, you will never attain knowledge except with six
things.\\
I will tell you about them clearly:\\
intelligence, eagerness, effort, provision,\\
companionship of a teacher, and a long time.''}\\
--- \textbf{Imām al-Shāfiʿī}, on the path to knowledge

\end{tcolorbox}

\begin{center}\rule{0.5\linewidth}{0.5pt}\end{center}

\begin{tcolorbox}[enhanced jigsaw, opacitybacktitle=0.6, colbacktitle=quarto-callout-note-color!10!white, arc=.35mm, opacityback=0, colback=white, rightrule=.15mm, breakable, titlerule=0mm, colframe=quarto-callout-note-color-frame, title=\textcolor{quarto-callout-note-color}{\faInfo}\hspace{0.5em}{Review Checklist}, leftrule=.75mm, left=2mm, bottomtitle=1mm, toptitle=1mm, bottomrule=.15mm, toprule=.15mm, coltitle=black]

\begin{itemize}
\tightlist
\item[$\square$]
  Master classroom objects and directions\\
\item[$\square$]
  Use present tense verbs in context\\
\item[$\square$]
  Build common study-related phrases\\
\item[$\square$]
  Practice dialogue with school vocabulary
\end{itemize}

\end{tcolorbox}

\section{الطَّعَام -- Food and Meals}\label{unit4-food}

In this chapter, we explore everything related to food and meals in
Arabic --- from ordering at a restaurant to naming common foods and
drinks. You'll also practice useful verbs and sentence structures for
daily eating routines.

\begin{center}\rule{0.5\linewidth}{0.5pt}\end{center}

\begin{tcolorbox}[enhanced jigsaw, opacitybacktitle=0.6, colbacktitle=quarto-callout-tip-color!10!white, arc=.35mm, opacityback=0, colback=white, rightrule=.15mm, breakable, titlerule=0mm, colframe=quarto-callout-tip-color-frame, title=\textcolor{quarto-callout-tip-color}{\faLightbulb}\hspace{0.5em}{Watch This!}, leftrule=.75mm, left=2mm, bottomtitle=1mm, toptitle=1mm, bottomrule=.15mm, toprule=.15mm, coltitle=black]

Need help remembering all this new vocabulary?\\
Check out this video:\\
\href{https://www.youtube.com/watch?v=TZYOScgVc3g&t}{How to Memorize
Arabic Vocabulary Fast!}

\end{tcolorbox}

\subsection{Key Vocabulary and
Phrases}\label{key-vocabulary-and-phrases-3}

\begin{longtable}[]{@{}ll@{}}
\toprule\noalign{}
English & Arabic \\
\midrule\noalign{}
\endhead
\bottomrule\noalign{}
\endlastfoot
Meal & وَجْبَة (ج) وَجَبَات \\
Breakfast & الْفَطُور \\
Lunch & الغَدَاء \\
Dinner & العَشَاء \\
Full, satisfied & شَبْعَان \\
Hungry & جَائِع \\
Thirsty & عَطْشَان / ظَمْآن \\
Sated (not thirsty) & رَيَّان \\
Before / After & قَبْل / بَعْد \\
Hot / Cold & سَاخِن / بَارِد \\
Fruits & فَاكِهَة (ج) فَوَاكِه \\
Vegetables & خُضَار \\
Chicken & دَجَاجَة (ج) دَجَاج \\
Fish & سَمَك \\
Bread & خُبْز \\
Meat & لَحْم (ج) لُحُوم \\
Rice & أُرْز \\
Eggs & بَيْض \\
Cheese & جُبْن \\
Yoghurt & زَبَادِي \\
Pickles & مُخَلَّل / طُرْشِي \\
Honey & عَسَل \\
Dessert & حَلْوَى \\
Falafel & فَلاَفِل / طَعْمِيَّة \\
Tea & شَاي \\
Coffee & قَهْوَة \\
Juice & عَصِير (ج) عَصَائِر \\
Milk & لَبَن / حَلِيب \\
Water & مَاء (ج) مِيَاه \\
Sugar & سُكَّر \\
Jam & مُرَبَّى \\
Butter & زُبْدَة \\
Olives & زَيْتُون \\
\end{longtable}

\begin{center}\rule{0.5\linewidth}{0.5pt}\end{center}

\subsection{At the Restaurant}\label{at-the-restaurant}

\begin{longtable}[]{@{}ll@{}}
\toprule\noalign{}
English & Arabic \\
\midrule\noalign{}
\endhead
\bottomrule\noalign{}
\endlastfoot
Restaurant & مَطْعَم \\
Waiter & نَادِل / عَامِل الْمَطْعَم \\
Menu & قَائِمَة الطَّعَام \\
Bill / Check & الحِسَاب \\
Change & البَاقِي \\
Customer & زَبُون \\
Reserved / Vacant & مَحْجُوز / خَالٍ \\
Table & مَائِدَة / طَاوِلَة \\
Glass / Cup & كُوب / فِنْجَان / كَأْس \\
Plate & طَبَق \\
Spoon / Fork / Knife & مِلْعَقَة / شَوْكَة / سِكِّين \\
Napkin & مِنْدِيل وَرَقِي \\
Water bottle & زُجَاجَة مَاء \\
Carbonated water & مِيَاه غَازِيَّة \\
Lemon juice & عَصِير لَيْمُون \\
Fried potatoes & بَطَاطِس مَقْلِيَّة \\
Kebab / Kofta & كَبَاب مَشْوِي / كُفْتَة \\
Shrimp & جَمْبَرِي \\
Soup & حَسَاء \\
\end{longtable}

\begin{center}\rule{0.5\linewidth}{0.5pt}\end{center}

\begin{tcolorbox}[enhanced jigsaw, opacitybacktitle=0.6, colbacktitle=quarto-callout-note-color!10!white, arc=.35mm, opacityback=0, colback=white, rightrule=.15mm, breakable, titlerule=0mm, colframe=quarto-callout-note-color-frame, title=\textcolor{quarto-callout-note-color}{\faInfo}\hspace{0.5em}{Try This!}, leftrule=.75mm, left=2mm, bottomtitle=1mm, toptitle=1mm, bottomrule=.15mm, toprule=.15mm, coltitle=black]

Translate the following using the vocabulary above:

\begin{enumerate}
\def\labelenumi{\arabic{enumi}.}
\tightlist
\item
  I want a chicken sandwich.\\
\item
  I am thirsty --- I want lemon juice.\\
\item
  Where is the restaurant?\\
\item
  The coffee is hot.\\
\item
  The salad is cold.
\end{enumerate}

\end{tcolorbox}

\subsection{Useful Verbs with Food}\label{useful-verbs-with-food}

\begin{longtable}[]{@{}ll@{}}
\toprule\noalign{}
Arabic Verb & English \\
\midrule\noalign{}
\endhead
\bottomrule\noalign{}
\endlastfoot
يَأْكُل & He eats \\
يَتَنَاوَل & He has / consumes (a meal) \\
يَشْرَب & He drinks \\
يُرِيد & He wants \\
يُنَادِي & He calls (the waiter) \\
تَطْلُب & You order / request \\
\end{longtable}

\subsubsection{Example Sentences:}\label{example-sentences}

\begin{itemize}
\item
  أَنَا أَتَنَاوَلُ الْفَطُورَ فِي السَّاعَةِ السَّابِعَةِ.\\
  \emph{I eat breakfast at 7:00.}
\item
  هُوَ يَشْرَبُ الشَّايَ وَالْقَهْوَةَ.\\
  \emph{He drinks tea and coffee.}
\item
  نُرِيدُ قَائِمَةَ الطَّعَامِ، مِنْ فَضْلِكَ.\\
  \emph{We would like the menu, please.}
\end{itemize}

\begin{center}\rule{0.5\linewidth}{0.5pt}\end{center}

\begin{tcolorbox}[enhanced jigsaw, opacitybacktitle=0.6, colbacktitle=quarto-callout-caution-color!10!white, arc=.35mm, opacityback=0, colback=white, rightrule=.15mm, breakable, titlerule=0mm, colframe=quarto-callout-caution-color-frame, title=\textcolor{quarto-callout-caution-color}{\faFire}\hspace{0.5em}{Click to Practice a Conversation}, leftrule=.75mm, left=2mm, bottomtitle=1mm, toptitle=1mm, bottomrule=.15mm, toprule=.15mm, coltitle=black]

\subsection{Dialogue: At the
Restaurant}\label{dialogue-at-the-restaurant}

\textbf{Waiter:} Welcome!\\
\textbf{Customer:} Thank you. Is this table free?\\
\textbf{Waiter:} No, it's reserved. But this one is free.\\
\textbf{Customer:} Thank you. Can I see the menu?\\
\textbf{Waiter:} Of course. Here you go.\\
\textbf{Customer:} I want grilled fish and lemon juice.\\
\textbf{Waiter:} Anything else?\\
\textbf{Customer:} No, thank you.\\
\textbf{Waiter:} The bill, sir?\\
\textbf{Customer:} Yes, here you go. Keep the change.\\
\textbf{Waiter:} Thank you very much!

\end{tcolorbox}

\begin{center}\rule{0.5\linewidth}{0.5pt}\end{center}

\subsection{Verb Practice with Singular
Pronouns}\label{verb-practice-with-singular-pronouns}

Let's look at \textbf{يَأْكُلُ} (he eats) and \textbf{يَشْرَبُ} (he drinks)
across the singular pronouns:

\begin{longtable}[]{@{}lll@{}}
\toprule\noalign{}
Pronoun & Eat = يَأْكُل & Drink = يَشْرَب \\
\midrule\noalign{}
\endhead
\bottomrule\noalign{}
\endlastfoot
أَنَا & آكُلُ & أَشْرَبُ \\
أَنْتَ & تَأْكُلُ & تَشْرَبُ \\
أَنْتِ & تَأْكُلِينَ & تَشْرَبِينَ \\
هُوَ & يَأْكُلُ & يَشْرَبُ \\
هِيَ & تَأْكُلُ & تَشْرَبُ \\
\end{longtable}

\begin{quote}
Note: These verbs are irregular in the past tense but regular in the
present.
\end{quote}

\begin{center}\rule{0.5\linewidth}{0.5pt}\end{center}

\begin{tcolorbox}[enhanced jigsaw, opacitybacktitle=0.6, colbacktitle=quarto-callout-note-color!10!white, arc=.35mm, opacityback=0, colback=white, rightrule=.15mm, breakable, titlerule=0mm, colframe=quarto-callout-note-color-frame, title=\textcolor{quarto-callout-note-color}{\faInfo}\hspace{0.5em}{Quick Questions}, leftrule=.75mm, left=2mm, bottomtitle=1mm, toptitle=1mm, bottomrule=.15mm, toprule=.15mm, coltitle=black]

Try answering in Arabic:

\begin{enumerate}
\def\labelenumi{\arabic{enumi}.}
\tightlist
\item
  مَاذَا تُرِيدُ لِلغَدَاءِ؟ -- What do you want for lunch?\\
\item
  هَلْ تُرِيدُ مَاءً أَوْ عَصِيرًا؟ -- Do you want water or juice?\\
\item
  أَيْنَ الْمَطْعَم؟ -- Where is the restaurant?\\
\item
  مَا هَذِهِ الْوَجْبَة؟ -- What is this meal?
\end{enumerate}

\end{tcolorbox}

\begin{center}\rule{0.5\linewidth}{0.5pt}\end{center}

\begin{tcolorbox}[enhanced jigsaw, opacitybacktitle=0.6, colbacktitle=quarto-callout-tip-color!10!white, arc=.35mm, opacityback=0, colback=white, rightrule=.15mm, breakable, titlerule=0mm, colframe=quarto-callout-tip-color-frame, title=\textcolor{quarto-callout-tip-color}{\faLightbulb}\hspace{0.5em}{A Scholar's Reflection}, leftrule=.75mm, left=2mm, bottomtitle=1mm, toptitle=1mm, bottomrule=.15mm, toprule=.15mm, coltitle=black]

\textbf{قَالَ الإِمَامُ الشَّافِعِيُّ:}\\
\emph{``مَا شَبِعْتُ مُنْذُ سِتَّةَ عَشَرَ سَنَةً، لِأَنَّ الشِّبَعَ يُثَقِّلُ الْبَدَنَ، وَيُقَسِّي الْقَلْبَ،
وَيُزِيلُ الْفِطْنَةَ، وَيَجْلِبُ النُّعَاسَ، وَيُضْعِفُ صَاحِبَهُ عَنِ الْعِبَادَةِ.''}

\emph{``I have not eaten to fullness for sixteen years, because satiety
burdens the body, hardens the heart, removes sharpness of mind, brings
on sleep, and weakens one from worship.''}

--- \textbf{Imām al-Shāfiʿī}

\end{tcolorbox}

\begin{center}\rule{0.5\linewidth}{0.5pt}\end{center}

\subsection{Quick Review Checklist}\label{quick-review-checklist}

\begin{tcolorbox}[enhanced jigsaw, opacitybacktitle=0.6, colbacktitle=quarto-callout-note-color!10!white, arc=.35mm, opacityback=0, colback=white, rightrule=.15mm, breakable, titlerule=0mm, colframe=quarto-callout-note-color-frame, title=\textcolor{quarto-callout-note-color}{\faInfo}\hspace{0.5em}{Review Goals}, leftrule=.75mm, left=2mm, bottomtitle=1mm, toptitle=1mm, bottomrule=.15mm, toprule=.15mm, coltitle=black]

\begin{itemize}
\tightlist
\item[$\square$]
  Use common food vocabulary\\
\item[$\square$]
  Order at a restaurant in Arabic\\
\item[$\square$]
  Conjugate ``eat'' and ``drink'' with pronouns\\
\item[$\square$]
  Ask for the bill, the menu, or water\\
\item[$\square$]
  Know useful utensils and table items\\
\item[$\square$]
  Practice a restaurant dialogue
\end{itemize}

\end{tcolorbox}

\section{الأوقات والأسعار - Times and
Prices}\label{ux627ux644ux623ux648ux642ux627ux62a-ux648ux627ux644ux623ux633ux639ux627ux631---times-and-prices}

Time and money shape much of our daily life and planning. In this unit,
you'll learn how to tell time, understand prices and currency, and talk
about schedules, appointments, and travel plans---all essential skills
for navigating everyday situations in Arabic.

\subsection{Key Vocabulary and
Phrases}\label{key-vocabulary-and-phrases-4}

\begin{longtable}[]{@{}ll@{}}
\toprule\noalign{}
English & Arabic \\
\midrule\noalign{}
\endhead
\bottomrule\noalign{}
\endlastfoot
Time, appointed time & مُوْعِد / وَقْت \\
It's 1 o'clock sharp & الوَاحِدَة تَمَامًا \\
Fifteen minutes to \ldots{} & إِلَّا رُبْعًا \\
Twenty minutes to \ldots{} & إِلَّا ثُلْثًا \\
Half past \ldots{} & وَالنِّصْف \\
Employee & مُوَظَّف \\
Ticket & تَذْكِرَة سَفَر \\
Airline office & مَكْتَب الطَّيَرَان \\
By day × By night & نَهَارًا × لَيْلًا \\
Quarter past \ldots{} & وَالرُّبْع \\
An hour - a minute - a second & سَاعَة - دَقِيقَة - ثَانِيَة \\
Airport & مَطَار \\
Plane, airplane & طَائِرَة \\
Train & قِطَار \\
Bus & حَافِلَة \\
Twenty past \ldots{} & وَالثُّلْث \\
New × Old & جَدِيد × قَدِيم \\
(She) goes & تَذْهَب \\
(He) sleeps & يَنَام \\
(She) arrives at & تَصِلُ \\
In the morning × In the evening & صَبَاحًا × مَسَاءً \\
The next × The previous & التَّالِي × السَّابِق \\
Before noon × In the afternoon & قَبْلَ الظُّهْر × بَعْدَ الظُّهْر \\
\end{longtable}

\subsubsection{Currency \& Exchange}\label{currency-exchange}

\begin{longtable}[]{@{}ll@{}}
\toprule\noalign{}
English & Arabic \\
\midrule\noalign{}
\endhead
\bottomrule\noalign{}
\endlastfoot
Late × Early & مُتَأَخِّر × مُبَكِّر \\
Exchange office & مَكْتَب الصِّرَافَة \\
Rate & سِعْر (ج) أَسْعَار \\
US dollar & دُوْلَار أَمْرِيكِي \\
Euro & يُورُو \\
Pound sterling & جُنَيْه إِنْجِلِيزِي \\
Egyptian pound & جُنَيْه مِصْرِي \\
Japanese yen & يِن يَابَانِي \\
Saudi riyal & رِيَال سَعُودِي \\
UAE dirham & دِرْهَم إِمَارَاتِي \\
Kuwaiti dinar & دِينَار كُوَيْتِي \\
Currency & عُمْلَة (ج) عُمْلَات \\
Money, cash & نُقُود \\
Turkish lira & لِيرَة تُرْكِيَّة \\
Change & نُقُود \\
\end{longtable}

\subsubsection{Banking \& Counting}\label{banking-counting}

\begin{longtable}[]{@{}ll@{}}
\toprule\noalign{}
English & Arabic \\
\midrule\noalign{}
\endhead
\bottomrule\noalign{}
\endlastfoot
(He) exchanges & يُبَدِّل \\
(He) counts & يَعُدُّ \\
Bank & بَنْك / مَصْرِف \\
Central bank & البَنْك المَرْكَزِي \\
Fifty piasters & نِصْف جُنَيْه \\
Twenty-five piasters & رُبْع جُنَيْه \\
\end{longtable}

\subsubsection{Prayer Times}\label{prayer-times}

\begin{longtable}[]{@{}ll@{}}
\toprule\noalign{}
English & Arabic \\
\midrule\noalign{}
\endhead
\bottomrule\noalign{}
\endlastfoot
Prayer Times & مَوَاقِيتُ الصَّلَاة \\
Adhan & الأَذَان \\
Dawn prayer & الفَجْر \\
Sunrise prayer & الشُّرُوق \\
Noon prayer & الظُّهْر \\
Afternoon prayer & العَصْر \\
Sunset prayer & المَغْرِب \\
Evening prayer & العِشَاء \\
\end{longtable}

\begin{center}\rule{0.5\linewidth}{0.5pt}\end{center}

\subsection{Translation Practice}\label{translation-practice}

\begin{tcolorbox}[enhanced jigsaw, opacitybacktitle=0.6, colbacktitle=quarto-callout-note-color!10!white, arc=.35mm, opacityback=0, colback=white, rightrule=.15mm, breakable, titlerule=0mm, colframe=quarto-callout-note-color-frame, title=\textcolor{quarto-callout-note-color}{\faInfo}\hspace{0.5em}{Translate the following into Arabic}, leftrule=.75mm, left=2mm, bottomtitle=1mm, toptitle=1mm, bottomrule=.15mm, toprule=.15mm, coltitle=black]

\begin{enumerate}
\def\labelenumi{\arabic{enumi}.}
\tightlist
\item
  I go to the airport at one o'clock sharp.
\item
  The bus arrives at a quarter past four.
\item
  He exchanges money at the bank.
\item
  The prayer times are written on the wall.
\item
  I sleep at night and go to work in the morning.
\item
  The price of the ticket is ten dollars.
\item
  I want to buy twenty-five piasters of change.
\end{enumerate}

\end{tcolorbox}

\begin{center}\rule{0.5\linewidth}{0.5pt}\end{center}

\subsection{Practice Conjugation}\label{practice-conjugation}

\begin{tcolorbox}[enhanced jigsaw, opacitybacktitle=0.6, colbacktitle=quarto-callout-tip-color!10!white, arc=.35mm, opacityback=0, colback=white, rightrule=.15mm, breakable, titlerule=0mm, colframe=quarto-callout-tip-color-frame, title=\textcolor{quarto-callout-tip-color}{\faLightbulb}\hspace{0.5em}{Verb Practice: Conjugate}, leftrule=.75mm, left=2mm, bottomtitle=1mm, toptitle=1mm, bottomrule=.15mm, toprule=.15mm, coltitle=black]

Practice conjugating these important verbs across pronouns:

\begin{itemize}
\tightlist
\item
  ذَهَبَ (to go)
\item
  نَامَ (to sleep)
\item
  وَصَلَ (to arrive)
\item
  بَدَّلَ (to exchange)
\item
  عَدَّ (to count)
\end{itemize}

Try writing full sentences using each form.

\end{tcolorbox}

\begin{center}\rule{0.5\linewidth}{0.5pt}\end{center}

\begin{tcolorbox}[enhanced jigsaw, opacitybacktitle=0.6, colbacktitle=quarto-callout-note-color!10!white, arc=.35mm, opacityback=0, colback=white, rightrule=.15mm, breakable, titlerule=0mm, colframe=quarto-callout-note-color-frame, title=\textcolor{quarto-callout-note-color}{\faInfo}\hspace{0.5em}{Review Checklist}, leftrule=.75mm, left=2mm, bottomtitle=1mm, toptitle=1mm, bottomrule=.15mm, toprule=.15mm, coltitle=black]

\begin{itemize}
\tightlist
\item[$\square$]
  Memorized all vocabulary in time, transport, and money
\item[$\square$]
  Can tell time using quarter/half/twenty expressions
\item[$\square$]
  Know names of prayer times
\item[$\square$]
  Can form basic sentences about time and prices
\item[$\square$]
  Can conjugate essential verbs from this chapter
\end{itemize}

\end{tcolorbox}




\end{document}
